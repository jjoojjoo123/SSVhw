\documentclass[11pt]{article}
\usepackage{xeCJK}
\usepackage{changepage}
\usepackage{bussproofs}
\usepackage{amsmath}
\usepackage{amssymb}
\usepackage{stmaryrd}
\usepackage[top=2cm,left=0.5cm,right=0.5cm,bottom=2cm]{geometry}

\renewcommand{\baselinestretch}{1.5}

\title{HW5}
\author{r08725027 林宏陽}
\date{\today}

\begin{document}
	\maketitle
	\begin{enumerate}
	\item 
		\begin{enumerate}
			\item ~
			\item By induction on $S$:
			\begin{itemize}
				\item Skip : $S\equiv$ \textbf{skip}
					\begin{align*}
						&\{wlp(\textbf{skip}, q)\}\textbf{ skip }\{q\}\\
						\leftrightarrow~ & \{q\}\textbf{ skip }\{q\}
					\end{align*}
					is valid since $\mathcal{M}\llbracket\textbf{skip}\rrbracket(q) = \emptyset \subseteq \llbracket q\rrbracket$.
				\item Assingment : $S\equiv u:=t$
					\begin{align*}
						&\{wlp(u:=t, q)\}~u:=t~\{q\}\\
						\leftrightarrow~ & \{q[t/u]\}~u:=t~\{q\}
					\end{align*}
					is valid by soundness.
				\item Composition : $S\equiv S_{1};S_{2}$\\
					From induction hypothesis, we have $\models \{wlp(S_{1}, q)\}~S_{1}~\{q\}$ and $\models \{wlp(S_{2}, q)\}~S_{2}~\{q\}$, that is,\\ $\mathcal{M}\llbracket S_{1} \rrbracket (\llbracket wlp(S_{1}, q) \rrbracket) \subseteq \llbracket q \rrbracket$ and $\mathcal{M}\llbracket S_{2} \rrbracket (\llbracket wlp(S_{2}, q) \rrbracket) \subseteq \llbracket q \rrbracket$.\\ To prove $\models \{wlp(S_{1};S_{2},q)\}~S_{1};S_{2}~\{q\}$, we need to show $\mathcal{M}\llbracket S_{1};S_{2}\rrbracket(\llbracket wlp(S_{1};S_{2}, q)\rrbracket) \subseteq \llbracket q \rrbracket$.
					\begin{align*}
						&\mathcal{M}\llbracket S_{1};S_{2}\rrbracket(\llbracket wlp(S_{1};S_{2}, q)\rrbracket)\\
						=~ & \mathcal{M}\llbracket S_{2}\rrbracket(\mathcal{M}\llbracket S_{1}\rrbracket(\llbracket wlp(S_{1}, wlp(S_{2}, q)) \rrbracket))\\
						&\{\text{Monotonicity of }\mathcal{M}\llbracket S_{2} \rrbracket\}\\
						\subseteq~ & \mathcal{M}\llbracket S_{2} \rrbracket(\llbracket wlp(S_{2}, q) \rrbracket)\\
						&\{\text{Induction Hypothesis}\}\\
						\subseteq~ & \llbracket q \rrbracket
					\end{align*}
				\item Conditional : $S\equiv$ \textbf{if} $B$ \textbf{then} $S_{1}$ \textbf{else} $S_{2}$ \textbf{fi}\\
					From induction hypothesis, we have $\models \{wlp(S_{1}, q)\}~S_{1}~\{q\}$ and $\models \{wlp(S_{2}, q)\}~S_{2}~\{q\}$, that is,\\ $\mathcal{M}\llbracket S_{1} \rrbracket (\llbracket wlp(S_{1}, q) \rrbracket) \subseteq \llbracket q \rrbracket$ and $\mathcal{M}\llbracket S_{2} \rrbracket (\llbracket wlp(S_{2}, q) \rrbracket) \subseteq \llbracket q \rrbracket$.\\ To prove $\models \{wlp(S,q)\}~S~\{q\}$, we need to show $\mathcal{M}\llbracket S\rrbracket(\llbracket wlp(S, q)\rrbracket) \subseteq \llbracket q \rrbracket$.
					\begin{align*}
						&\mathcal{M}\llbracket S\rrbracket(\llbracket wlp(S, q)\rrbracket)\\
						=~ & \mathcal{M}\llbracket S_{1}\rrbracket(\llbracket (B \land wlp(S_{1}, q)) \lor (\neg B \land wlp(S_{2}, q))\rrbracket \cap \llbracket B \rrbracket)\\
						& \cup \mathcal{M}\llbracket S_{2}\rrbracket(\llbracket (B \land wlp(S_{1}, q)) \lor (\neg B \land wlp(S_{2}, q))\rrbracket \cap \llbracket \neg B \rrbracket)\\
						=~ & \mathcal{M}\llbracket S_{1}\rrbracket(\llbracket (B \land wlp(S_{1}, q)) \rrbracket) \cup \mathcal{M}\llbracket S_{2}\rrbracket(\llbracket (\neg B \land wlp(S_{2}, q)) \rrbracket)\\
						&\{B \land wlp(S_{1}, q) \to wlp(S_{1}, q) \text{ and } \neg B \land wlp(S_{2}, q) \to wlp(S_{2}, q)\}\\
						\subseteq~ & \mathcal{M}\llbracket S_{1}\rrbracket(\llbracket (wlp(S_{1}, q)) \rrbracket) \cup \mathcal{M}\llbracket S_{2}\rrbracket(\llbracket (wlp(S_{2}, q)) \rrbracket)\\
						&\{\text{Induction Hypothesis}\}\\
						\subseteq~ & \llbracket q \rrbracket
					\end{align*}
				\item While : $S\equiv$ \textbf{while} $B$ \textbf{do} $S_{1}$ \textbf{od}\\
					From induction hypothesis, we have $\models \{wlp(S_{1}, q)\}~S_{1}~\{q\}$, that is, $\mathcal{M}\llbracket S_{1} \rrbracket (\llbracket wlp(S_{1}, q) \rrbracket) \subseteq \llbracket q \rrbracket$.\\ To prove $\models \{wlp(S,q)\}~S~\{q\}$, we need to show $\mathcal{M}\llbracket S\rrbracket(\llbracket wlp(S, q)\rrbracket) \subseteq \llbracket q \rrbracket$.
			\end{itemize}
		\end{enumerate}

	\item 
		\begin{enumerate}
			\item ~
			\item ~
		\end{enumerate}
	\item ~
		\begin{prooftree}

					\AxiomC{pred. calculus}
					\UnaryInfC{$x>y\land x<y \to y=x$}

					\AxiomC{}
					\RightLabel{(Assign.)}
					\UnaryInfC{$\{z=x[y/z]\}~z:=y~\{z=x\}$}

				\RightLabel{(S.Pre)}
				\BinaryInfC{$\{x>y\land x<y\}~z:=y~\{z=x\}$}

					\AxiomC{$\alpha$}
			\RightLabel{(Conditional)}
			\BinaryInfC{$\{x>y\}$ \textbf{if} $x<y$ \textbf{then} $z:=y$ \textbf{else} $z:=x$ $\{z=x\}$}
			\RightLabel{(Proc.)}
			\UnaryInfC{$\{(x>y)[a,b/x,y]\}~max(a,b,c)~\{(z=x)[a,b,c/x,y,z]\}$}
		\end{prooftree}
		where $\alpha$ is the following tree:
		\begin{prooftree}
					\AxiomC{pred. calculus}
					\UnaryInfC{$x>y\land x\nless y \to x=x$}

					\AxiomC{}
					\RightLabel{(Assign.)}
					\UnaryInfC{$\{z=x[x/z]\}~z:=x~\{z=x\}$}
				\RightLabel{(S.Pre)}
				\BinaryInfC{$\{x>y\land x\nless y\}~z:=y~\{z=x\}$}
		\end{prooftree}
	\end{enumerate}
\end{document}