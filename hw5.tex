\documentclass[11pt]{article}
\usepackage{xeCJK}
\usepackage{changepage}
\usepackage{bussproofs}
\usepackage{amsmath}
\usepackage{amssymb}
\usepackage{stmaryrd}
\usepackage[top=2cm,left=0.5cm,right=0.5cm,bottom=2cm]{geometry}

\renewcommand{\baselinestretch}{1.5}

\title{HW5}
\author{r08725027 林宏陽}
\date{\today}

\begin{document}
	\maketitle
	\begin{enumerate}
	\item 
		\begin{enumerate}
			\item By definition of $\models \{p\}S\{q\}$, proving $\models \{p\}S\{q\}$ iff $p \to wlp(S,q)$ is equivalent to showing $\mathcal{M}\llbracket S \rrbracket(\llbracket p \rrbracket) \subseteq \llbracket q \rrbracket$ iff $\llbracket p \rrbracket \subseteq \llbracket wlp(S, q) \rrbracket$.\\
			Prove by induction on $S$:
			\begin{itemize}
				\item Skip : $S\equiv$ \textbf{skip}
					\begin{align*}
						& \mathcal{M}\llbracket \textbf{skip} \rrbracket(\llbracket p \rrbracket) = \llbracket p \rrbracket \subseteq \llbracket q \rrbracket \text{ iff }  p \to q \equiv p \to wlp(\textbf{skip},q) 
					\end{align*}
				\item Assingment : $S\equiv u:=t$
				\begin{enumerate}
					\item $\llbracket p \rrbracket \subseteq \llbracket q[t/u] \rrbracket \to \mathcal{M}\llbracket u:=t \rrbracket(\llbracket p \rrbracket) \subseteq \llbracket q \rrbracket$
					\begin{align*}
						& \mathcal{M}\llbracket u:=t \rrbracket(\llbracket p \rrbracket)\\
						&\{\text{Monotonicity of $\mathcal{M}\llbracket S \rrbracket$}\}\\
						\subseteq~ & \mathcal{M}\llbracket u:=t \rrbracket(\llbracket q[t/u] \rrbracket)\\
						&\{\text{Soundness proof}\}\\
						\subseteq~ & \llbracket q \rrbracket
					\end{align*}
					\item $\mathcal{M}\llbracket u:=t \rrbracket(\llbracket p \rrbracket) \subseteq \llbracket q \rrbracket \to \llbracket p \rrbracket \subseteq \llbracket q[t/u] \rrbracket$ is similar to i.
				\end{enumerate}
				\item Composition : $S\equiv S_{1};S_{2}$
					\begin{align*}
						& \llbracket p \rrbracket \subseteq \llbracket wlp(S_{1}, wlp(S_{2}, q)) \rrbracket\\
						&\{\text{Induction Hypothesis}\}\\
						\Leftrightarrow ~& \mathcal{M}\llbracket S_{1} \rrbracket(\llbracket p \rrbracket) \subseteq \llbracket wlp(S_{2}, q) \rrbracket\\
						&\{\text{Induction Hypothesis}\}\\
						\Leftrightarrow ~& \mathcal{M}\llbracket S_{2} \rrbracket(\mathcal{M} \llbracket S_{1} \rrbracket (\llbracket p \rrbracket)) \subseteq \llbracket q \rrbracket\\
						\Leftrightarrow ~& \mathcal{M} \llbracket S_{1};S_{2} \rrbracket(\llbracket p \rrbracket) \subseteq \llbracket q \rrbracket\\
					\end{align*}
				\item Conditional : $S\equiv$ \textbf{if} $B$ \textbf{then} $S_{1}$ \textbf{else} $S_{2}$ \textbf{fi}
					\begin{align*}
						& \llbracket p \rrbracket \subseteq \llbracket ((B \land wlp(S_{1}, q)) \lor (\neg B \land wlp(S_{2}, q)))\\
						\Leftrightarrow ~& \begin{cases}
							\llbracket p \land B \rrbracket \subseteq \llbracket B \land wlp(S_{1}, q) \rrbracket \subseteq \llbracket wlp(S_{1}, q) \rrbracket\\
							\llbracket p \land \neg B \rrbracket \subseteq \llbracket \neg B \land wlp(S_{2}, q) \rrbracket \subseteq \llbracket wlp(S_{2}, q) \rrbracket
						\end{cases}\\
						&\{\text{Induction Hypothesis}\}\\
						\Leftrightarrow ~& \begin{cases}
							\mathcal{M}\llbracket S_{1} \rrbracket (\llbracket p \land B \rrbracket) \subseteq \llbracket q \rrbracket\\
							\mathcal{M}\llbracket S_{2} \rrbracket (\llbracket p \land \neg B \rrbracket) \subseteq \llbracket q \rrbracket\\
						\end{cases}\\
						\Leftrightarrow ~& \mathcal{M}\llbracket S_{1} \rrbracket (\llbracket p \land B \rrbracket) \cup \mathcal{M}\llbracket S_{2} \rrbracket (\llbracket p \land \neg B \rrbracket) \subseteq \llbracket q \rrbracket\\
						\Leftrightarrow ~& \mathcal{M}\llbracket S_{1} \rrbracket (\llbracket p \rrbracket \cap \llbracket B \rrbracket) \cup \mathcal{M}\llbracket S_{2} \rrbracket (\llbracket p \rrbracket \cap \llbracket \neg B \rrbracket) \subseteq \llbracket q \rrbracket\\
						\Leftrightarrow ~& \mathcal{M} \llbracket S \rrbracket(\llbracket p \rrbracket) \subseteq \llbracket q \rrbracket
					\end{align*}
				\item While : $S\equiv$ \textbf{while} $B$ \textbf{do} $S_{1}$ \textbf{od}\\
					\begin{align*}
						& \bigcup_{k=0}^{\infty} \mathcal{M} \llbracket S^{k} \rrbracket(\llbracket p \rrbracket) \subseteq \llbracket q \rrbracket \to \llbracket p \rrbracket \subseteq \llbracket wlp(S, q) \rrbracket
					\end{align*}
					The base case $k=0$:
					\begin{align*}
						& \mathcal{M} \llbracket S^{0} \rrbracket(\llbracket p \rrbracket) = \emptyset \subseteq \llbracket q \rrbracket \leftrightarrow  \llbracket p \rrbracket \subseteq \llbracket wlp(S^{0}, q) \rrbracket
					\end{align*}
					Inductive steps:
					\begin{align*}
						& \mathcal{M} \llbracket S^{k+1} \rrbracket(\llbracket p \rrbracket) \subseteq \llbracket q \rrbracket\\
						\Leftrightarrow ~& \mathcal{M} \llbracket \textbf{if } B \textbf{ then } S_{1};S^{k} \textbf{ else skip fi} \rrbracket(\llbracket p \rrbracket) \subseteq \llbracket q \rrbracket\\
						\Leftrightarrow ~& \mathcal{M} \llbracket S^{k}\rrbracket(\mathcal{M} \llbracket S_{1}\rrbracket(\llbracket p \land B \rrbracket)) \cup \mathcal{M} \llbracket \textbf{skip} \rrbracket (\llbracket p \land \neg B \rrbracket) \subseteq \llbracket q \rrbracket\\
						\Leftrightarrow ~& \begin{cases}
							\mathcal{M} \llbracket S^{k}\rrbracket(\mathcal{M} \llbracket S_{1}\rrbracket(\llbracket p \land B \rrbracket)) \subseteq \llbracket q \rrbracket\\
							\mathcal{M} \llbracket \textbf{skip} \rrbracket (\llbracket p \land \neg B \rrbracket) \subseteq \llbracket q \rrbracket
						\end{cases}
					\end{align*}
					By Induction Hypothesis, we have:
					\begin{align*}
						\begin{cases}
							\mathcal{M}\llbracket S_{1} \rrbracket(\llbracket p \land B \rrbracket) \subseteq \llbracket wlp(S^{k}, q) \rrbracket\\
							\llbracket p \land \neg B \rrbracket \subseteq \llbracket wlp(\textbf{skip}, q) \rrbracket = \llbracket q \rrbracket
						\end{cases}
					\end{align*}
					Again, we have:
					\begin{align*}
						\begin{cases}
							\llbracket p \land B \rrbracket \subseteq \llbracket wlp(S_{1}, wlp(S^{k}, q)) \rrbracket = \llbracket wlp(S^{k+1}, q) \rrbracket\\
							\llbracket p \land \neg B \rrbracket \subseteq \llbracket wlp(\textbf{skip}, q) \rrbracket = \llbracket q \rrbracket
						\end{cases}
					\end{align*}
					Finally, 
					\begin{align*}
						& \llbracket p \rrbracket = \llbracket p \land B \rrbracket \cup \llbracket p \land \neg B \rrbracket\\
						\subseteq ~& \llbracket wlp(S^{k+1}, q) \rrbracket \cup \llbracket q \rrbracket
					\end{align*}
			\end{itemize}


			\item By induction on $S$:
			\begin{itemize}
				\item Skip : $S\equiv$ \textbf{skip}
					\begin{align*}
						&\{wlp(\textbf{skip}, q)\}\textbf{ skip }\{q\}\\
						\leftrightarrow~ & \{q\}\textbf{ skip }\{q\}
					\end{align*}
					is valid since $\mathcal{M}\llbracket\textbf{skip}\rrbracket(q) = \emptyset \subseteq \llbracket q\rrbracket$.
				\item Assingment : $S\equiv u:=t$
					\begin{align*}
						&\{wlp(u:=t, q)\}~u:=t~\{q\}\\
						\leftrightarrow~ & \{q[t/u]\}~u:=t~\{q\}
					\end{align*}
					is valid by soundness.
				\item Composition : $S\equiv S_{1};S_{2}$\\
					From induction hypothesis, we have $\models \{wlp(S_{1}, q)\}~S_{1}~\{q\}$ and $\models \{wlp(S_{2}, q)\}~S_{2}~\{q\}$, that is,\\ $\mathcal{M}\llbracket S_{1} \rrbracket (\llbracket wlp(S_{1}, q) \rrbracket) \subseteq \llbracket q \rrbracket$ and $\mathcal{M}\llbracket S_{2} \rrbracket (\llbracket wlp(S_{2}, q) \rrbracket) \subseteq \llbracket q \rrbracket$.\\ To prove $\models \{wlp(S_{1};S_{2},q)\}~S_{1};S_{2}~\{q\}$, we need to show $\mathcal{M}\llbracket S_{1};S_{2}\rrbracket(\llbracket wlp(S_{1};S_{2}, q)\rrbracket) \subseteq \llbracket q \rrbracket$.
					\begin{align*}
						&\mathcal{M}\llbracket S_{1};S_{2}\rrbracket(\llbracket wlp(S_{1};S_{2}, q)\rrbracket)\\
						=~ & \mathcal{M}\llbracket S_{2}\rrbracket(\mathcal{M}\llbracket S_{1}\rrbracket(\llbracket wlp(S_{1}, wlp(S_{2}, q)) \rrbracket))\\
						&\{\text{Monotonicity of }\mathcal{M}\llbracket S_{2} \rrbracket\}\\
						\subseteq~ & \mathcal{M}\llbracket S_{2} \rrbracket(\llbracket wlp(S_{2}, q) \rrbracket)\\
						&\{\text{Induction Hypothesis}\}\\
						\subseteq~ & \llbracket q \rrbracket
					\end{align*}
				\item Conditional : $S\equiv$ \textbf{if} $B$ \textbf{then} $S_{1}$ \textbf{else} $S_{2}$ \textbf{fi}\\
					From induction hypothesis, we have $\models \{wlp(S_{1}, q)\}~S_{1}~\{q\}$ and $\models \{wlp(S_{2}, q)\}~S_{2}~\{q\}$, that is,\\ $\mathcal{M}\llbracket S_{1} \rrbracket (\llbracket wlp(S_{1}, q) \rrbracket) \subseteq \llbracket q \rrbracket$ and $\mathcal{M}\llbracket S_{2} \rrbracket (\llbracket wlp(S_{2}, q) \rrbracket) \subseteq \llbracket q \rrbracket$.\\ To prove $\models \{wlp(S,q)\}~S~\{q\}$, we need to show $\mathcal{M}\llbracket S\rrbracket(\llbracket wlp(S, q)\rrbracket) \subseteq \llbracket q \rrbracket$.
					\begin{align*}
						&\mathcal{M}\llbracket S\rrbracket(\llbracket wlp(S, q)\rrbracket)\\
						=~ & \mathcal{M}\llbracket S_{1}\rrbracket(\llbracket (B \land wlp(S_{1}, q)) \lor (\neg B \land wlp(S_{2}, q))\rrbracket \cap \llbracket B \rrbracket)\\
						& \cup \mathcal{M}\llbracket S_{2}\rrbracket(\llbracket (B \land wlp(S_{1}, q)) \lor (\neg B \land wlp(S_{2}, q))\rrbracket \cap \llbracket \neg B \rrbracket)\\
						=~ & \mathcal{M}\llbracket S_{1}\rrbracket(\llbracket (B \land wlp(S_{1}, q)) \rrbracket) \cup \mathcal{M}\llbracket S_{2}\rrbracket(\llbracket (\neg B \land wlp(S_{2}, q)) \rrbracket)\\
						&\{B \land wlp(S_{1}, q) \to wlp(S_{1}, q) \text{ and } \neg B \land wlp(S_{2}, q) \to wlp(S_{2}, q)\}\\
						\subseteq~ & \mathcal{M}\llbracket S_{1}\rrbracket(\llbracket (wlp(S_{1}, q)) \rrbracket) \cup \mathcal{M}\llbracket S_{2}\rrbracket(\llbracket (wlp(S_{2}, q)) \rrbracket)\\
						&\{\text{Induction Hypothesis}\}\\
						\subseteq~ & \llbracket q \rrbracket
					\end{align*}
				\item While : $S\equiv$ \textbf{while} $B$ \textbf{do} $S_{1}$ \textbf{od}\\
					From induction hypothesis, we have $\models \{wlp(S_{1}, q)\}~S_{1}~\{q\}$, that is, $\mathcal{M}\llbracket S_{1} \rrbracket (\llbracket wlp(S_{1}, q) \rrbracket) \subseteq \llbracket q \rrbracket$.\\ To prove $\models \{wlp(S,q)\}~S~\{q\}$, we need to show $\mathcal{M}\llbracket S\rrbracket(\llbracket wlp(S, q)\rrbracket) \subseteq \llbracket q \rrbracket$.
					\begin{align*}
						&\mathcal{M}\llbracket S\rrbracket(\llbracket wlp(S, q)\rrbracket)\\
						=~ & \bigcup_{k=0}^{\infty}\mathcal{M}\llbracket S^{k} \rrbracket (\llbracket wlp(S,q) \rrbracket)
					\end{align*}
						We prove by induction that, for all $k \geq 0$, $\mathcal{M}\llbracket S^{k} \rrbracket (\llbracket wlp(S,q) \rrbracket) \subseteq \llbracket q \rrbracket$\\
						The base case $k=0$:
					\begin{align*}
						&\mathcal{M}\llbracket S^{0}\rrbracket(\llbracket wlp(S,q) \rrbracket) = \emptyset \subseteq \llbracket q \rrbracket
					\end{align*}
						Inductive steps:
					\begin{align*}
						&\mathcal{M}\llbracket S^{k+1}\rrbracket(\llbracket wlp(S,q) \rrbracket)\\
						=~ & \mathcal{M}\llbracket \textbf{if } B \textbf{ then } S_{1}; S^{k} \textbf{ else skip fi}\rrbracket(\llbracket wlp(S,q) \rrbracket)\\
						=~ & \mathcal{M}\llbracket S^{k} \rrbracket (\mathcal{M}\llbracket S_{1} \rrbracket (wlp(S, q) \land B) \rrbracket) \cup \llbracket wlp(S,q) \land \neg B \rrbracket\\
						&\{\text{Lemma 5 for wlp on the slide}\}\\
						\subseteq~ & \mathcal{M}\llbracket S^{k} \rrbracket (\mathcal{M}\llbracket S_{1} \rrbracket (wlp(S_{1}, wlp(S, q))) \rrbracket) \cup \llbracket q \rrbracket\\
						&\{\text{Induction Hypothesis and Monotonicity}\}\\
						\subseteq~ & \mathcal{M}\llbracket S^{k} \rrbracket (\mathcal{M}\llbracket S_{1} \rrbracket (wlp(S, q)) \rrbracket) \cup \llbracket q \rrbracket\\
						&\{\text{Induction Hypothesis}\}\\
						\subseteq~ & \llbracket q \rrbracket \cup \llbracket q \rrbracket = \llbracket q \rrbracket
					\end{align*}
			\end{itemize}
		\end{enumerate}

	\item 
		\begin{enumerate}
			% \item By induction on $S$:
			% \begin{itemize}
			% 	\item Skip : $S\equiv$ \textbf{skip}\\
			% 		If $Q_{1} \Rightarrow Q_{2}$ then $wp(S, Q_{1}) \Rightarrow wp(S, Q_{2}) \triangleq Q_{1} \Rightarrow Q_{2}$ holds.
			% 	\item Assingment : $S\equiv u:=t$\\
			% 		If $Q_{1} \Rightarrow Q_{2}$ then $wp(S, Q_{1}) \Rightarrow wp(S, Q_{2}) \triangleq Q_{1}[t/u] \Rightarrow Q_{2}[t/u]$ holds.
			% 	\item Composition : $S\equiv S_{1};S_{2}$\\
			% 		From induction hypothesis, we have if $Q_{1} \Rightarrow Q_{2}$ then $wp(S_{1}, Q_{1}) \Rightarrow wp(S_{1}, Q_{2})$ and $wp(S_{2}, Q_{1}) \Rightarrow wp(S_{2}, Q_{2})$\\
			% 		If $Q_{1} \Rightarrow Q_{2}$ then
			% 		\begin{align*}
			% 			& wp(S, Q_{1}) \Rightarrow wp(S, Q_{2})\\
			% 			\triangleq ~& wp(S_{1}, wp(S_{2}, Q_{1})) \Rightarrow wp(S_{1}, wp(S_{2}, Q_{2}))
			% 		\end{align*}
			% 		holds by induction.
			% 	\item Conditional :
			% 		\begin{align*}
			% 			S\equiv &~ \textbf{if } B_{1} \to S_{1}\\
			% 			&~ [] B_{2} \to S_{2}\\
			% 			&~ \cdots\\
			% 			&~ [] B_{n} \to S_{n}\\
			% 			&~ \textbf{fi}
			% 		\end{align*}
					
			% 		From induction hypothesis, we have if $Q_{1} \Rightarrow Q_{2}$ then $wp(S_{i}, Q_{1}) \Rightarrow wp(S_{i}, Q_{2})$ for all $i$ from $1$ to $n$.\\
			% 		If $Q_{1} \Rightarrow Q_{2}$ then
			% 		\begin{align*}
			% 			& wp(S, Q_{1}) \Rightarrow wp(S, Q_{2})\\
			% 			\triangleq ~& (\exists i:1 \leq i \leq n : B_{i}) \land (\forall i:1 \leq i \leq n : B_{i} \to wp(S_{i}, Q_{1}))\\
			% 			& \Rightarrow (\exists i:1 \leq i \leq n : B_{i}) \land (\forall i:1 \leq i \leq n : B_{i} \to wp(S_{i}, Q_{2}))\\
			% 			\triangleq ~& \forall i:1 \leq i \leq n : B_{i} \to wp(S_{i}, Q_{1})\Rightarrow \forall i:1 \leq i \leq n : B_{i} \to wp(S_{i}, Q_{2})\\
			% 		\end{align*}
			% \end{itemize}
			
			\item ~
			\begin{align*}
				& wp(S,Q_{1})\\
				& \{(Q_{1} \Rightarrow Q_{2}) \Rightarrow (Q_{1} \equiv Q_{1} \land Q_{2})\}\\
				\equiv ~& wp(S, Q_{1}\land Q_{2})\\
				\equiv ~& wp(S,Q_{1}) \land wp(S,Q_{2})\\
				\Rightarrow ~& wp(S,Q_{2})
			\end{align*}
			\item To prove $wp(S, Q_{1}) \lor wp(S, Q_{2}) \Rightarrow wp(S, Q_{1} \lor Q_{2})$, we need
				\begin{enumerate}
					\item $wp(S, Q_{1}) \Rightarrow wp(S, Q_{1} \lor Q_{2})$ and
					\item $wp(S, Q_{2}) \Rightarrow wp(S, Q_{1} \lor Q_{2})$
				\end{enumerate}
				The proof of i. and ii. can be done by applying Law of Monotonicity to $Q_{1} \Rightarrow Q_{1} \lor Q_{2}$ and $Q_{2} \Rightarrow Q_{1} \lor Q_{2}$ respectively.
		\end{enumerate}
	\item ~
		\begin{prooftree}

					\AxiomC{pred. calculus}
					\UnaryInfC{$x>y\land x<y \to y=x$}

					\AxiomC{}
					\RightLabel{(Assign.)}
					\UnaryInfC{$\{z=x[y/z]\}~z:=y~\{z=x\}$}

				\RightLabel{(S.Pre)}
				\BinaryInfC{$\{x>y\land x<y\}~z:=y~\{z=x\}$}

					\AxiomC{$\alpha$}
			\RightLabel{(Conditional)}
			\BinaryInfC{$\{x>y\}$ \textbf{if} $x<y$ \textbf{then} $z:=y$ \textbf{else} $z:=x$ $\{z=x\}$}
			\RightLabel{(Proc.)}
			\UnaryInfC{$\{(x>y)[a,b/x,y]\}~max(a,b,c)~\{(z=x)[a,b,c/x,y,z]\}$}
		\end{prooftree}
		where $\alpha$ is the following tree:
		\begin{prooftree}
					\AxiomC{pred. calculus}
					\UnaryInfC{$x>y\land x\nless y \to x=x$}

					\AxiomC{}
					\RightLabel{(Assign.)}
					\UnaryInfC{$\{z=x[x/z]\}~z:=x~\{z=x\}$}
				\RightLabel{(S.Pre)}
				\BinaryInfC{$\{x>y\land x\nless y\}~z:=y~\{z=x\}$}
		\end{prooftree}
	\end{enumerate}
\end{document}